\documentclass{article}
\usepackage[left=1.5cm, right=2cm]{geometry}

\title{Computer Science Project}
\author{Zakariyya Kurmally (2315839), Ihsaan Ramjanee (2315007), Nirvesh Bhagirath (2313628)}
\date{\today}



\begin{document}
\maketitle
\tableofcontents
\pagebreak
\section{Rules of Cartesia}
Cartesia is a 2D platformer. The aim to the game is to collect
gold enough gold coins to unlock the chest at the bottom of the 
map. You will need to avoid traps, monsters and falling into the void. 

\begin{itemize}
    \item WASD or arrow keys can be used to move the character
    \item Player starts with 3 lives
    \item Falling into a trap or off the map results in death
\end{itemize}

\section{Function description}
To understand the working of the following functions, it will be important 
to know the contents of the struct GameState. The latter contains information 
needed for these functions to work properly. Here are its components and their
description:

\begin{center}

    \begin{tabular} {|c|c|}
        \hline
        Texture2D tileset[NUM\textunderscore OF\textunderscore TEXTURES]; & Array of textures \\
        \hline
        Texture2D bg1; & Background image\\
        \hline
        Particle particles[PARTICLES\textunderscore COUNT]; & Array of particles\\
        \hline
        int card[MAP\textunderscore SIZE][MAP\textunderscore SIZE]; & Card array representing entities\\
        \hline
        int collisionMap[MAP \textunderscore SIZE][MAP\textunderscore SIZE]; & Array indicating where collisions should occur \\
        \hline
        int playerPos[2]; & Contains the coordinates of the player\\
        \hline
        int coinsCollected; & ...\\
        \hline
        int remainingLife;  & ...\\
        \hline
        Sound sounds[2]; & Array of sounds used \\
        \hline
        Direction direction; & Direction the player is facing\\
        \hline
        int playerMoving; & Indicates if the player is moving\\
        \hline
        Texture2D scarfy; & Spritesheet of player character\\
        \hline
        int framesCounter; & \\
        int framesSpeed; & Used to handle sprite animation\\
        int currentFrame; & \\
        Rectangle frameRec; &\\
        \hline
    \end{tabular}
    
\end{center}

\subsection{utils.c}
Contains utility functions mostly for handling graphics and in calculations. 
\begin{itemize}
    \item \textunderscore initGraphics (): Starts the window and audio devices
    \item \textunderscore processImage (const char* filepath, Rectangle src
    Vector2 size): Loads image files from disk, resizes them, converts them
    to the proper data type and returns them
    \item \textunderscore loadResources(GameState* currGameState): loads all the images and
    sounds from disk into memory. Makes use of \textunderscore processImage
    \item \textunderscore unloadResources(GameState* currGameState): unloads all the loaded 
    images from memory
    \item \textunderscore updateFrameInfo(GameState* currGameState): used to keep track 
    of frame times for character animation
    \item \textunderscore initParticles(GameState* currGameState): creates particles
    with random values for position and speed
    \item \textunderscore displayParticles(GameState* currGameState): 
        outputs the particles screen
    \item \textunderscore updateParticles(GameState* currGameState): 
        updates the position of 
    particles based on the speed ans re-positions them if need be
\end{itemize}

\subsection{core.c}
Contains the main functions of the game. Used for displaying graphics and initialising
the map.
\begin{itemize}
    \item \textunderscore init \textunderscore card (GameState* currGameState): fills the map 
    with entities like platforms, traps, trees. It is called before the 
    main game loop
    \item \textunderscore display \textunderscore card (GameState* currGameState): loops through
    the array card, located inside currGameState, and displays the appropriate textures
    \item \textunderscore handle \textunderscore player (GameState* currGameState): handles player movement,
    collisions and player respawns
\end{itemize}

\section{Problems encountered and how they were dealt with}
\subsection{VCS}
When working on such a project, we would often need to switch back to previous versions of
the code after a certain implementation went south. It was becoming a pain to remember what 
was changed and how to revert back. This was solved using git which allowed changes to be commited 
and easy reversion.

\subsection{Unscalable function parameters}
Many functions of Cartesia needed access to distinct information about the game state and 
textures. Early on, we noticed that passing them one by one would have been an absolute
headache, especially when unloading all the resources since we make use of more than 10 of them. 
As such, we created the GameState struct which holds all the necessary information for the proper
functioning of our functions.

\section{Limitations}
\subsection{Not cross-platform (Yet)}
C does not statically link with Raylib by default. As such, the executable is not distributable.
However, shipping the DLL (for Windows) or the SO (for Linux) along with the compiled binary 
should solve this problem. Furthermore, Raylib has build scripts for Android.

\subsection{No save points}
Apart from loading in necessary game resources, Cartesia does not yet have functionality to
save or load a game from a file on disk.

\section{Individual Contributions}
\subsection {Bhagirath Nirvesh (25\%)}
Handled the particles system, sound and the display of textures.
\subsection {Ihsaan Ramjanee (35\%)}
Map initialisation, collision detection and texture placement.
\subsection {Zakariyya Kurmally (40\%)}
Graphics processing, player movement, library setup as well as structuring the project.

    
\end{document}
